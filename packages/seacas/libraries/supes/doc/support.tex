\chapter{SUPPORT PROGRAMMER'S GUIDE} \label{sec:support}
This chapter documents the internal architecture for SUPES.  It is intended
to guide the maintenance of SUPES and support of SUPES on new operating
systems.  The consequences of using the Skeleton FORTRAN extension library
on the internal operation of SUPES is fully discussed.


\section{Free Field Input} The SUPES free field input system consists of
four subroutines: FREFLD (section~\ref{sec:frefld}),
FFISTR (section~\ref{sec:ffistr}), GETINP
(section~\ref{sec:getinp}), and STRIPB (section~\ref{sec:stripb}). All of
these routines are written in fully standard ANSI FORTRAN.

FREFLD calls the extension library routine EXUPCS
(section~\ref{sec:exupcs}). If only the skeleton version of EXUPCS is
available, case insensitivity of input data (rule~\ref{itm:case} of
section~\ref{sec:syntax}) can not be guaranteed.

FFISTR is the input line parsing routine.
It is called by FREFLD,
but the user is free to call it independently.
The input line may be of arbitrary length.

GETINP calls the extension library routine EXREAD
(section~\ref{sec:exread}). If only the skeleton version of EXREAD is
available, GETINP will not prompt nor guarantee echo when reading from the
standard input device (KIN = 0).


\subsection{Implementation Notes on FREFLD}

This section contains a basic outline of the internal operation of the free
field input system and other supplemental information.  More complete
documentation is contained within the code itself.

FREFLD is organized into five phases:
\begin{enumerate}
\item All the output arrays are initialized to their default values.

\item The next input record is obtained via GETINP.  Processing of a
        continuation line begins with this phase.

\item The effective portion of the input line is isolated by stripping any
        comment and leading/trailing blanks.  A flag is set if a
        continuation line is to follow this record.

\item All field separators are made uniform.  This phase streamlines the
        main processing loop which follows.

\item Successive fields are extracted, translated, and categorized until
        the input line is exhausted.  After the maximum number of fields is
        reached, fields are counted but not processed further.
\end{enumerate}
Upon leaving the main translation loop, the routine is restarted at phase 2
if the continuation flag is set.

The only errors returned by FREFLD are any returned from GETINP.

A data field is left-justified to define a CHARACTER value, but must be
right-justified to obtain a numeric value.  An internal READ is used to
decode a numeric value from a data field.  FREFLD relies upon the IOSTAT
specifier to determine if the field represents a valid numeric format; this
presents the possibility that some non-standard numeric strings may be
interpreted inconsistently by various operating systems.  Default numeric
values are overwritten if and only if IOSTAT indicates a valid translation.

CHARACTER data manipulation tends to be the area of lowest reliability for
FORTRAN compilers, especially with supercomputers.  An attempt was made in
coding FREFLD to minimize the risk of triggering compiler bugs by
manipulating pointers rather than shifting CHARACTER strings.


\subsection{Test Program for FREFLD}
A simple test program which calls FREFLD is included with the SUPES free
field input system.  FREFLD is instructed to digest data entered via the
standard input device (e.g., keyboard), then the results are dumped to the
standard output device (e.g., screen).  This program should always be run to
verify proper operation of FREFLD on a new operating system or compiler.
Application programmers are encouraged to experiment with this program to
learn what to expect from FREFLD.
A sample session from a Sun 4/60 Workstation follows:
\begin{verbatim}

% ffrtest         <-- At the system prompt, enter the program name.
    1: This is an example <-- At the SUPES prompt, the user enters a line, etc.
 NFIELD =    4
    I     KV(I)               CV(I)          RV(I)       IV(I)
    1         0  "THIS            "         0.               0
    2         2  "IS              "         0.               0
    3         0  "AN              "         0.               0
    4         0  "EXAMPLE         "         0.               0
    5        -1  "                "         0.               0
   2: Another line = example.
 NFIELD =    3
    I     KV(I)               CV(I)          RV(I)       IV(I)
    1         0  "ANOTHER         "         0.               0
    2         0  "LINE            "         0.               0
    3         0  "EXAMPLE.        "         0.               0
    4        -1  "                "         0.               0
    5        -1  "                "         0.               0
   3: This is a further 3.e5
 NFIELD =    6
    I     KV(I)               CV(I)          RV(I)       IV(I)
    1         0  "THIS            "         0.               0
    2         2  "IS              "         0.               0
    3         2  "A               "         0.               0
    4         0  "FURTHER         "         0.               0
    5         2  "3.E5            "      3.000E+05      300000
   4: exit
 NFIELD =    1
    I     KV(I)               CV(I)          RV(I)       IV(I)
    1         0  "EXIT            "         0.               0
    2        -1  "                "         0.               0
    3        -1  "                "         0.               0
    4        -1  "                "         0.               0
    5        -1  "                "         0.               0
   5: ^C                         <-- To exit, the user enters a ^C.
%
\end{verbatim}

\section{Memory Manager} \label{sec:table}
This section includes details of the internal operations of the memory
manager, assumptions used in the memory manager, and details on the
implementation of the memory manager on systems which do not support the
extension library.


\subsection{Table Architecture and Maintenance}
The bookkeeping for the memory manager is accomplished with three tables; a
memory block table, a void area table, and a dictionary.

The {\em
memory block table} maintains a record of contiguous blocks of memory
that have been received from the operating system.  If a series of requests
causes separate blocks to become contiguous, these blocks are joined.  The
beginning location and length of each memory block is recorded, and the
table is sorted in location order.

Within each memory block, sections of memory that are not currently
allocated to arrays are recorded in the {\em void area table}.  As in the
case of the memory block table, contiguous voids are joined and this table
is sorted in location order.

The {\em dictionary} relates storage locations with eight character array
names. The dictionary is sorted via the default FORTRAN collating sequence.
All characters (including blanks) are significant.
All names are converted to upper case then
blank filled or truncated to eight characters.
In addition to the array name, the
dictionary stores the location and length of each dynamic array.

Any call for memory (MDGET or MDRSRV) will be satisfied in one of two ways:
\begin{enumerate}
\item If a void of sufficient size is available, then this void will be
          used for the new array (MDRSRV).  In the case of MDGET, no further
          action is taken.

\item An extension library call (EXMEMY) is made to get more memory from
          the system.
\end{enumerate}

A request to extend an array (MDLONG) is satisfied in one of three ways:
\begin{enumerate}
\item If a void of sufficient size exists at the end of the array, then
          this space is allocated to the array.

\item If a void large enough for the extended array exists elsewhere in
          memory, the array is moved to this location.  Note that the data
          is actually shifted and the pointer is updated.

\item An extension library call (EXMEMY) is made to get more memory from
          the system.
\end{enumerate}

A call to MDCOMP will cause all arrays within each memory block to be moved
to the lower addresses (pointers) within that memory block.  Thus, all voids
in the block will be joined at the end of the block.

A call to MDGIVE will attempt to return memory to the system.  Only voids at
the end of a memory block are subject to this attempt, and the system may
accept only portions of these.  Thus a call to MDCOMP followed by MDGIVE
will release the maximum memory to the system.


\subsection{Non-ANSI FORTRAN Assumptions}
Although the memory manager is written in standard FORTRAN-77, it does
depend on some assumptions which are not part of the ANSI standard.  These
assumptions are:
\begin{enumerate}
\item The contents of a word are not checked nor altered by an INTEGER
          assignment.  Data is moved by MDLONG or MDCOMP as INTEGER
          variables.

\item Strong typing is not enforced between dummy and actual arguments.
          This allows the same base array to pass storage to any INTEGER,
          REAL, or LOGICAL array.

\item Array bounds are not enforced.  Thus, any value is a valid
          subscript for the base array.

\item All dynamically allocated memory must remain fixed in relation to
          the base array.
\end{enumerate}

\subsection{Standard FORTRAN Implementation}

If an installation does not yet support the extension library, it is still
possible and advantageous to use the memory manager.  In this case, the
memory manager will act as a dynamic allocator of static (already
dimensioned) memory.  Codes which employ the memory manager therefore do not
need to be rewritten, and codes under development can anticipate the
implementation of the extension library.

When the subprograms IXLNUM or EXMEMY of the extension library are not
available, the following steps must be taken before using the memory
manager:
\begin{enumerate}
\item Install the skeleton version of the extension library
(Section~\ref{sec:skel}).

\item Alter the memory manager subroutine MDINIT as follows:

\underline{ORIGINAL}

\verb+DIMENSION MYV(1)+

\underline{ALTERED}
\begin{verbatim}
PARAMETER (MAXSIZ=1024)
COMMON /EXTLIB/ MYV( MAXSIZ)
\end{verbatim}

\item Put the base vector in the user's program in the COMMON block
          EXTLIB and dimension it consistently with the COMMON blocks in
          EXMEMY and MDINIT.

\item If more than 1024 numeric storage units are required, change the
          parameter statement in MDINIT, EXMEMY and the user's program.
\end{enumerate}

\subsection{Test Program}
In order to aid the installation of the memory manager at a new site, an
interactive test program has been written which allows the user to exercise
each of the features of the memory manager and insure that it is operating
properly.
While the proper implementation of the memory management test program
requires an in-depth examination of the corresponding source file,
a short test run on a Cray running the UNICOS operating system follows
(comments are included after an arrow, \verb+<--+):
\begin{verbatim}
% memtest    <-- At the system prompt, enter program name.
FUNC: mdinit    <-- At the SUPES prompt, the user enters a string, etc.
FUNC: mcinit
FUNC: mdwait
FUNC: mdrsrv real1 108
 POINTER:    -65733
FUNC: mcrsrv char 850
 POINTER:   -532696
FUNC: mdrsrv real2 108
 POINTER:    -65733
FUNC: mdexec
 POINTER BEFORE     -65733
 POINTER AFTER      17879   <-- Having the pointer updated is vital!
FUNC: mdlist


 **************************************************
0 * * * * * * *  D I C T I O N A R Y  * * * * * * *
0                               NUMERIC   CHARACTER
	 NAME      LOCATION      LENGTH      LENGTH

    1  CHAR           17664         107         850
    2  REAL1          17771         108          -1
    3  REAL2          17879         108          -1
0        TOTAL                      323         850


0 * * * V O I D   T A B L E * * *
0           LOCATION       LENGTH

      1        17987           61
0        TOTAL                 61


 **************************************************
0 * * * * * *  O R D E R E D   L I S T  * * * * * *
0                               NUMERIC   CHARACTER
	 NAME      LOCATION      LENGTH      LENGTH

    1  CHAR           17664         107         850
    2  REAL1          17771         108          -1
    3  REAL2          17879         108          -1
    4                 17987          61
 BLOCK SIZE                         384         850

 ALLOCATED TOTAL                    384         850

     GRAND TOTAL                    384         850

FUNC: exit
   STOP              in MEMTEST
%
\end{verbatim}


\section{Extension Library Implementation}
Implementing the SUPES extension library on a new operating system
requires a firm understanding of that system, but should not require a great
deal of programming.  Since the package is by definition system dependent,
it is impossible to predict the exact procedure which will be required to
implement these routines on a given operating system.  This section provides
some general guidelines and hints compiled from experience in implementing
the package on several very different systems.

As has been mentioned previously,
this version represents a change in philosophy regarding the procedure
for implementing a port of the extension library.
Specifically,
many of the features of the extension library require a
richer data type than is available in ANSI FORTRAN 77.
For example,
the requirement to do pointer assignment for the memory management
made it desirable to utilize a more flexible programming
language.
The language chosen was C\@.
A direct consequence of this is that the
entire SUPES extension library is now coded in a single
set of source files across all supported machines.
Among the advantages are:
\begin{enumerate}

\item It reduces the amount of bookkeeping that is necessary to maintain
      the library across a number of machine architectures at a given
      site,

\item It now allows for a codified approach
      to building the library on any given machine, and finally,

\item It permits one to use the current source as an example for a
      future port.

\end{enumerate}
Of course,
these advantages do come at a cost.
The FORTRAN--C interface must now be handled at the source level
in the extension library.
This is an extremely system dependent area.
However,
most systems do allow for such a scenario,
and,
accordingly,
it tends to be documented quite extensively.

The code should be well
commented and references to appropriate system manuals should be included.

The original FORTRAN version of the skeleton library will
continue to be part of the distribution.
To use them,
it is recommended that one start with the skeleton library routines and
gradually add system dependent code to provide full capability.
Examples are provided in the distribution in the form of
the original source code for VAX/VMS and Cray CTSS with the CFTLIB
FORTRAN run-time library.
In fact,
for the latter case that code represents the only method of implementation
of SUPES.
This is entirely due to the lack of a suitable FORTRAN-C interfacing scheme
under that system.

If this is the desired plan,
it is suggested that extension library modules be implemented in the
following order:
\begin{enumerate}
\item EXUPCS.  The skeleton version should be sufficient.

\item EXTIME, EXDATE, EXCPUS, IXLNUM, and EXPARM.  These routines are
        generally straightforward and can be accomplished simply with the
        aid of the FORTRAN and/or C Run Time Library manuals for the
        particular operating system.

\item EXREAD, EXNAME, EXMEMY, and IXLCHR.  These routines require a more intimate
        knowledge of the operating system.  A substantial set of system
        documentation may be required to accomplish these tasks.
\end{enumerate}

\subsection{Implementation Notes for Modules}
The format of the date for \underline{EXDATE} must be strictly observed.
Many systems supply a date service routine which formats the date in a
different style. Conversion to the SUPES format should be straightforward.

Most systems provide a time of day service routine which formats the time in
the desired style.  Some systems also return fractional seconds which can
easily be trimmed off.  In any case, the format specified by
\underline{EXTIME} must be strictly observed.

\underline{EXCPUS} is intended to measure performance rather than cost.  The
quantity returned by EXCPUS should be raw CPU seconds; any weighting for
memory use or priority should be removed.  I/O time should be included only
if it is performed by the CPU.

The hardware ID string for \underline{EXPARM} should reflect both the
manufacturer and model of the processor.  For example, ``VAX 8600'' rather
than just ``VAX'' allows the user to make sense of the CPU time returned by
EXCPUS.

The software ID string should reflect the release of the operating system in
use, such as ``COS 1.11''.  It is not a trivial exercise to provide all
pertinent information in eight characters for ad hoc systems like CTSS which
vary widely between installations.  For example, the string ``CFTLIB14'' has
been used to indicate a variation of the SUPES package for CTSS using CFTLIB
and the CFT 1.14 compiler.

On most systems KCSU will give the number of characters per numeric word and
KNSU will be unity.  For a hypothetical 36-bit processor which allows 8-bit
characters to cross word boundaries, KCSU=9 and KNSU=2 would define the
storage relationship.

The proper value for IDAU should always be indicated in the reference manual
for the compiler where it discusses Unformatted Direct Access files.

The unit/file mode of \underline{EXNAME} should follow as closely as
possible to whatever convention the particular operating system uses for
connecting a FORTRAN I/O unit to a file at execution time.  This feature
should be easy to implement on systems which support pre-connection.  Support
for units 1-99 should be sufficient.

The symbol mode feature of EXNAME should be designed to obtain messages from
the system level procedure which activates the program.  Eight characters
per symbol is a reasonable limit.  Support for symbols 0-7 should be
adequate.

Support for EXNAME not only requires coding the routine itself, but also
designing the system procedure level interface.  This interface should
always be designed before coding EXNAME.  It should fit as cleanly as
possible into normal techniques for writing procedures for the system.

The skeleton version of \underline{EXUPCS} is designed to work on any system
which supports lower case letters.  This routine will rarely require any
change.

\underline{EXREAD} must provide a prompt for an interactive device and
guarantee that input is echoed.  This requires a careful determination of
the current execution environment.  For example, EXREAD must be able to
handle input from a script file as well as from a terminal.  Any automatic
echo service provided by the operating system should be employed wherever
possible, as long as the user supplied prompt appears along with the input
data echo.

In all instances,
the C programming language provides a clean method for returning the
address for \underline{IXLNUM}\@. In some cases it may be necessary to
convert the address to numeric units.  For example, addresses on VMS must be
divided by four to convert from bytes to numeric storage units.
The same cannot necessarily be said for a character address
as returned by \underline{IXLCHR}\@.
The reader is referred to the source file \verb+ixlchr.c+ for further details
on how to attack this problem.

\underline{EXMEMY} is the most crucial routine in the extension
library---and one the primary reasons for choosing to do the extension
library in C.
As opposed to in the past,
this latest approach has made it one of the most straightforward in
the entire extension library.
However, care should still be
taken to ensure that both memory block locations and sizes are measured in
numeric storage units.
In the current version of SUPES,
memory is allocated in blocks of 512 bytes
(a number which can be changed at compile time)
to improve performance.
EXMEMY should return the precise amount of
memory allocated.
Any memory that is given by the system,
but not requested by the user is kept track of in a void table
by the memory manager.
So, it is generally unnecessary to keep track of memory
blocks allocated via EXMEMY.

\subsection{Extension Library Test Program}
A short program which exercises all features of the SUPES extension
library is available.  This program should be considered a starting point
for testing a new implementation.  Other tests which more extensively
exercise complex modules, such as EXMEMY, should be developed as needed.
An example session on a Sun 4/60 Workstation follows (with comments offset by
an arrow, \verb+<--+):
\begin{verbatim}

% setenv FOR001 junk.dat  <-- Test EXNAME.
% exttest        <-- At the system prompt, invoke the procedure.
TST: ldkj        <-- At the SUPES prompt, the user enters a string.
Input line = LDKJ    <-- The input line is returned in upper case.
Date = 12/18/89
Time = 09:58:05
Unit 1 name = junk.dat
Unit 10 name =
Symbol 1 =
Processor = Sun4      System = OS4.0.3c  Mode =   1
Character, Numeric, D/A Units:   4  1  0
Memory block location and length:   24700  128
Numeric difference =   4
Character difference =   4
CPU time =     7.00000E-02

\end{verbatim}

\section{Installation Documentation Guidelines}
A supplement to this document should be written for each operating system on
which SUPES is installed.  As a minimum, this supplement should include:
\begin{enumerate}
\item How to access the SUPES library and link it to an applications
        program.  Individual copies of SUPES should never be propagated as
        this reduces the quality assurance level of SUPES.

\item How to interface from the operating system to EXNAME for both unit/
        file mode and symbol mode.

\item How to interface to EXREAD via an interactive device.  Information
        such as how to signal an end of file should be specified.
\end{enumerate}

Any known bugs or idiosyncrasies.

The installation supplements for several operating systems are included in
the Appendix.
